\documentclass[margin,line]{res}
\usepackage{helvet}
\renewcommand{\familydefault}{\sfdefault}
\renewcommand{\sectionfont}


\addtolength{\topmargin}{-.2in}
\addtolength{\textheight}{1.2in}
\oddsidemargin -.25in
\evensidemargin -.25in
\newsectionwidth{1in}
\textwidth=5.9in
\sectionwidth=1.in
% if using pdflatex:
%\setlength{\pdfpagewidth}{\paperwidth}
%\setlength{\pdfpageheight}{\paperheight} 

\newenvironment{list1}{
  \begin{list}{\ding{113}}{%
      \setlength{\itemsep}{0in}
      \setlength{\parsep}{0in} \setlength{\parskip}{0in}
      \setlength{\topsep}{0in} \setlength{\partopsep}{0in} 
      \setlength{\leftmargin}{0.17in}}}{\end{list}}
\newenvironment{list2}{
  \begin{list}{$\bullet$}{%
      \setlength{\itemsep}{0in}
      \setlength{\parsep}{0in} \setlength{\parskip}{0in}
      \setlength{\topsep}{0in} \setlength{\partopsep}{0in} 
      \setlength{\leftmargin}{0.2in}}}{\end{list}}



\begin{document}
\pagenumbering{gobble}
\name{Jordan Eli Rossen }

\begin{resume}
\section{Contact\\ Information}
jordanerossen@gmail.com             \hfill 319-541-0640 \\
github.com/jordanero
\section{Education}
{\bf Harvard University}  \hfill 2020 - 2025 \\
PhD in Epidemiology - Statistical genetics\\
PhD advisor: Alkes Price\\\\
{\bf Harvard University}  \hfill 2020 - 2025 \\
M.S. in Biostatistics \\\\
{\bf Tufts University}  \hfill 2013 - 2017 \\
B.S. with majors in Computer Science and Chemistry\\
Major GPA: 3.95, GPA: 3.89\\
\section{Professional \\Experience}
{\bf Harvard University} \hfill Alkes Price Group\\
\vspace{-.3cm}\\
PhD Researcher \hfill 2020-2025\\
\begin{list2}
\item Developed MultiSuSiE, an open source Python package for multi-ancestry fine-mapping with higher power and lower computational cost than alternative methods (Rossen et al. 2024 bioRxiv). 
\item Studied differences in heritability components across ancestries using hundreds of thousands of whole genome sequences from All of Us (Rossen and Price 2024 American Society of Human Genetics Conference).
\end{list2}
{\bf Broad Institute of MIT and Harvard} \hfill Cancer Data Science Group\\
\vspace{-.3cm}\\
Associate Computational Biologist II \hfill 2019 - 2020\\
Associate Computational Biologist I \hfill 2017 - 2019\\
\begin{list2}
%\item Developed data processing methodology for, led assay development for, and mined publicly accessible datasets for large, high-throughput, public functional genomic (DepMap Project Achilles, Dempster et al. 2019, BioRxiv) and drug screening datasets (DepMap PRISM Repurposing, Corsello et al. 2020 Nature Cancer, Tsvetkov et al. 2022 Science).
\item Developed data processing methodology for DepMap Achilles, a public, high-throughput CRISPR-Cas9 genome-scale knockout dataset (Dempster et al. 2019 BioRxiv, $>$1,000 cell lines x 20,000 genes).
\item Lead assay development analyst for PRISM, a high-throughput, in-vitro drug screening platform (Corsello et al. 2020 Nature Cancer, $>$500 cell lines x 4,500 drugs).
\item Analyzed public high-throughput drug screening datasets to identify promising small molecules (Tsvetkov et al. 2019 Nature Chemical Biology, Tsvetkov et al. 2022 Science).
\end{list2}
{\bf Pfizer Inc.} \hfill Precision Medicine\\
\vspace{-.3cm}\\
Molecular Data  Intern \hfill Summer 2016\\
\begin{list2}
\item Designed and implemented an internal metabolomics biomarker database using RShiny. 
\end{list2}
{\bf University of Illinois}\hfill Kami Hull Organometallic Catalysis Group\\
\vspace{-.3cm}\\
REU researcher \hfill Summer 2015\\
\begin{list2}
\item Expanded the reaction scope of ex-situ, Pd-catalyzed, chloroform-based alkoxycarbonylation reactions.
\end{list2}
{\bf Tufts University}\hfill Lin Computational Chemistry Group\\
\vspace{-.3cm}\\
Research Assistant \hfill  2013 - 2014\\
\begin{list2}
\item Programmed molecular dynamics simulations of liquid argon, oxygen, and water using Fortran.
\end{list2}
\newpage
\section{Teaching \\ Assistant}
{\bf Harvard University} \hfill Department of Epidemiology\\
\vspace{-.3cm}\\
Causal Inference (EPI207), Jamie Robbins \hfill Fall 2022, Fall 2023\\
Advanced Population and Medical Genetics  (EPI511), Alkes Price \hfill Spring 2023\\\\
{\bf Tufts University} \hfill Department of Computer Science\\
\vspace{-.3cm}\\
Algorithms (COMP160), Gregory Aloupis \hfill Fall 2017, Spring 2017
\section{Awards} 
{\bf Platform Talk} - American Society for Human Genetics \hfill 2023\\
{\bf F31 Grant} - National Institutes of Health \hfill  2023\\
{\bf Spot Award} - Broad Institute, Level 2 \hfill  2019\\
{\bf Deans List} - Tufts University \hfill  2013 - 2017\\
{\bf Snyder Scholarship} - UIUC, Department of Organic Chemistry \hfill 2015\\
{\bf Iowa State Policy Debate Tournament} - 2nd place \hfill  2013\\
\section{Programming} 
\begin{list2}
\item Proficiency in R and Python
\item Experience in C, C++, and Matlab
\end{list2}
\section{Publications} 
Cited more than 3,700 times across six publications
\\\\
Rossen, J., Shi, H., Strober, B., Zhang, M.J., Kanai, M., McCaw, Z.R., Liang, L., Weissbrod, O., Price, A.L., 2024. MultiSuSiE improves multi-ancestry fine-mapping in All of Us whole-genome sequencing data. medRxiv.\\\\
Strober, B.J., Zhang, M.J., Amariuta, T., Rossen, J., Price, A.L., 2024, Fine-mapping causal tissues and genes at disease-associated loci. in press at Nature Genetics.\\\\
Tsvetkov, P., Coy, S., Petrova, B., Dreishpoon, M., Verma, A., Abdusamad, M., Rossen, J., Joesch-Cohen, L., Humeidi, R., Spangler, R.D. and Eaton, J.K., Frenkel, E., Kocak, M., Corsello, S.M., Lutsekno, S., Kanarek, NM., Santagata, S., Golub, T.R., 2022. Copper induces cell death by targeting lipoylated TCA cycle proteins. Science, 375(6586), pp.1254-1261.\\\\
Dempster, J.M., Rossen, J., Kazachkova, M., Pan, J., Kugener, G., Root, D.E. and Tsherniak, A., 2019. Extracting biological insights from the project achilles genome-scale CRISPR screens in cancer cell lines. BioRxiv, p.720243.\\\\
Corsello, S.M., Nagari, R.T., Spangler, R.D., Rossen, J., [33 others], Golub, T.R., 2020. Discovering the anticancer potential of non-oncology drugs by systematic viability profiling. Nature Cancer, 1(2), pp.235-248.\\\\
Tsvetkov, P., Detappe, A., Cai, K., Keys, H.R., Brune, Z., Ying, W., Thiru, P., Reidy, M., Kugener, G., Rossen, J., Kockac, M., Kory, N., Tsherniak, A., Santagata, S., Whitesell, Luke., Ghobrial, I.M., Markley, J.L., Lindquist,m S., Golub, T.R., 2019. Mitochondrial metabolism promotes adaptation to proteotoxic stress. Nature Chemical Biology, 15(7), pp.681-689.
\end{resume}
\end{document}




